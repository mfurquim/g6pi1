\chapter{Aquisição de Software MPS.BR}

\begin{center}
	Este documento visa orientar a aquisição de software para automatização do projeto de Casa Sustentável e Automatizada
	\end{center}

%---------------------------------------------------------------%
%-------------------- Objetivo da Aquisição --------------------%
%---------------------------------------------------------------%
\section{Objetivo da Aquisição}

\subsection{Justificativa}

	Para o amplo conhecimento das possíveis soluções é necessário uma pesquisa de todas as soluções já feitas e analizar
	 o custo do desenvolvimento de uma nova solução. Após esta pesquisa é necessário uma comparação.

	A princípio foram feitos os requisitos da solução que se encontra na seção~\ref{sec:requisitos} e o objetivo que
	detalha o pacote onde se encontra na sub-seção~\ref{subsec:objetivo}, que para generalização foi chamado de SACS (Sistema de Automação Para Casa Sustentável). A
	partir dessa análise de requisitos várias soluções deveriam ser apresentadas, tendo em mente que o desenvolvimento
	a partir do zero do conjunto de software e hardware seria uma opção inviável, dado o conjunto de gastos envolvento
	todo o projeto. Após ter sido feita uma busca pela internet por soluções de software existentes no
	mercado, uma tabela foi feita comparando quais destas soluções encontradas atendiam os requisitos especificados.
	Destes pacotes de software foi escolhido um que atendia todos os requisitos necessários do problema.

	Dado essas condições, é necessário justificar o porquê deste pacote escolhido teria um melhor custo-benefício
	comparado a uma solução de software desenvolvida do zero. Observando os requisitos, seria necessário um conjunto
	de softwares, muitos deles sendo críticos, como o sistema de seguraça, levando assim a um desenvolvimento
	 que seria muito longo e caro, o que tornaria o projeto inviável, além de não ser justificado pelo fato que as
	 soluções separadas para uma casa sustentável são áreas do conhecimento brandamente exploradas anteriormente por
	 outros indivíduos e instituições. Portanto foi concluido que desenvolver a solução ocasionaria um certo uso de
	 tempo e recursos que o projeto não prevê, além de estar ignorando o fato que já existem soluções amplamente
	 testadas e aprovadas pela comunidade.


\subsection{Objetivo}
\label{subsec:objetivo}

\subsubsection{Finalidade}

	Esta seção tem como objetivo apresentar, analisar e definir as necessidades e características do SACS. O
	SACS tem como objetivo controlar e monitorar todo o sistema do Projeto “Casa Sustentável”. A seção
	descreverá as características gerais do sistema, os envolvidos em seu desenvolvimento e o público alvo.

\subsubsection{Posicionamento}

\subsubsubsection{Descrição do Problema}

\begin{longtable}{|l|m{7cm}|}
	\hline \textbf{O problema} & Valor elevado do consumo desnecessário de energia elétrica\\
	\hline \textbf{Afeta} & Pessoas que moram em residências, sejam casas ou apartamentos, e fazem uso da energia
	elétrica\\
	\hline \textbf{Cujo impacto é} & Alto custo da conta de energia elétrica\\
	\hline \textbf{Uma boa solução seria} & Desenvolvimento de um sistema de automação capaz de controlar o consumo
	da energia elétrica de acordo com o requisitos.\\
	\hline
\caption{Descrição do problema}
\label{Descrição_do_problema}
\end{longtable}

\subsubsubsection{Sentença de Posição do Sistema}

\begin{longtable}{|l|m{7cm}|}
	\hline \textbf{Para} & Cidadãos brasileiros residentes na Casa Sustentável e Inteligente \\
	\hline \textbf{Que} & Se importam com o alto consumo desnecessário de energia elétrica e desejam maior conforto\\
	\hline \textbf{O SACS} & É um software ou um conjuntos de softwares de transmissão e controle de dados que
	integra sensores e age de forma inteligente\\
	\hline \textbf{Que} & Garante conforto para os residentes e economia de energia\\
	\hline \textbf{Diferente de} & Crestron\footnote{http://www.crestron.com/}\\
	\hline \textbf{Nosso Sistema} & Integra sensores de forma inteligente para controle do consumo de energia elétrica,
	armazenamento e reuso de água não potável\\
	\hline
\caption{Sentença de Posição do Sistema}
\label{Sentenca_de_Posicao_do_Sistema}
\end{longtable}

\subsubsection{Descrições dos Envolvidos e Usuários}

\subsubsubsection{Resumo dos Envolvidos}

\begin{longtable}{|m{5cm}|m{5cm}|m{5cm}|}
	\hline \textbf{Nome} & \textbf{Descrição} & \textbf{Responsabilidades}\\
	\hline Professor das disciplina de Projeto Integrador 1. & É responsável por ministrar aulas de projeto integrador
	1, direcionando caminhos para o projeto. & Mantem o foco do projeto ministrando aulas sobre gerência de projetos,
	direcionando o caminho a ser seguido pelo projeto. \\
	\hline Alunos de  Projeto Integrador 1. & São os alunos que cursam a disciplina de  Projeto Integrador 1. & Iram
	desenvolver o projeto, especificando quais tecnologias serão usadas, que se encaixam melhor no escopo do projeto,
	visando a sustentabilidade.\\
	\hline Aluno Gerente de Projeto Integrador 1. & É o aluno que cursa a disciplina de projeto Integrador e assumiu a
	posição de gerente. & Irá garantir que todos do grupo estão desempenhando sua função e deverá integrar todas as
	partes e passar para o gerente do projeto “Casa Sustentável”.\\
	\hline Desenvolvedores e Gerentes & Profissionais da área que implementarão e integrarão o projeto já pronto dos
	alunos de PI1. & Irão produzir o sistema visionado e projetado na matéria de PI1.\\
	\hline Moradores da “Casa Sustentável” & Usuário das funcionalidades da “Casa Sustentável” & Possuir fundo
	financeiro necessário para investir no projeto da “casa sustentável”\\
	\hline
\caption{Resumo dos Envolvidos}
\label{Resumo_dos_Envolvidos}
\end{longtable}

\subsubsubsection{Resumo dos Usuários}

\begin{longtable}{|m{5cm}|m{10cm}|}
	\hline \textbf{Nome} & \textbf{Descrição}\\
	\hline Moradores da “Casa Sustentável” & Pessoa Física que adquiriu o produto “Casa Sustentável”, especificado no
	projeto da disciplina de PI1.\\
	\hline
\caption{Resumo dos Usuários}
\label{Resumo_dos_Usuarios}
\end{longtable}

\subsubsubsection{Perfis dos Envolvidos}

\textit{Equipe projetista da disciplina de Pi1}

\begin{longtable}{|m{5cm}|m{10cm}|}
	\hline \textbf{Representantes} & Arnoldo T. M. Lima\\ & Flavio C. Paixao\\ & Itiane T. B. Almeida\\ & Lucas V. T.
	Brilhante\\ & Marcelo M. Oliveira\\ & Mateus M. F. Mendon\c{c}a\\ & Talyta V. Cabral\\ & Victor B. Batalha\\ &
	Gabriel S. Soares\\ & Vitor N. A. Ribeiro\\
	\hline \textbf{Descrição} & Farão o documento do projeto com todas as especificações técnicas necessárias.\\
	\hline \textbf{Tipo} & Estudantes da Universidade de Brasília, Campus do Gama da matéria de Projeto Integrador 1,
	que pesquisaram todas as especificações do projeto e desenvolveram usando todos os conhecimentos de projeto passado
	pelos professores.\\
	\hline \textbf{Responsabilidades} & Trabalhar em conjunto para montar um documento que especifique todos os aspectos
	 do projeto. Pesquisar conhecimentos técnicos a serem aplicados no projeto.\\
	\hline \textbf{Critério de Sucesso} & Gerir de maneira adequada, trabalhar em conjunto, desenvolver em equipe
	adequando ao tema do projeto, custo e prazo definido pela matéria.\\
	\hline \textbf{Envolvimento} & Alto.\\
	\hline
\caption{Perfis dos Envolvidos}
\label{Perfis_dos_Envolvidos}
\end{longtable}

\textit{Gerente de automação do projeto “Casa Sustentável”}

\begin{longtable}{|m{5cm}|m{10cm}|}
	\hline \textbf{Representantes} & Gabriel S. Soares\\
	\hline \textbf{Descrição} & Aluno da matéria de PI1 que é gerente da área de automação.\\
	\hline \textbf{Tipo} & Aluno habilitado e disponível, capaz de gerir a equipe de projeto de automação.\\
	\hline \textbf{Responsabilidades} & Garantir que todos os membros estão desempenhando sua função e entregando o que
	 foi decidido. Deve também organizar as entregas de forma a integrar os assuntos e garantir a qualidade do projeto.\\
	\hline \textbf{Critério de Sucesso} & Gerir corretamente o tempo de forma a abranger a diferentes variáveis que
	podem atrasar a entrega.\\
	\hline \textbf{Envolvimento} & Alto.\\
	\hline
\caption{Gerente de automação do projeto}
\label{Gerente_de_automacao_do_projeto}
\end{longtable}

\textit{Gerente de software do projeto “Casa Sustentável”}

\begin{longtable}{|m{5cm}|m{10cm}|}
	\hline \textbf{Representantes} & Vitor N. A. Ribeiro\\
	\hline \textbf{Descrição} & Aluno da matéria de PI1 que é gerente da área de software.\\
	\hline \textbf{Tipo} & Aluno habilitado e disponível, capaz de gerir a equipe de projeto de software.\\
	\hline \textbf{Responsabilidades} & Garantir que todos os membros estão desempenhando sua função e entregando o que
	 foi decidido. Garantir que a documentação e o processo de aquisição de software para automação seja corretamente
	 impregada e que satisfaça todos os requisitos necessários para a "Casa Sustentável". Deve também organizar as
	 entregas de forma a integrar os assuntos e garantir a qualidade do projeto.\\
	\hline \textbf{Critério de Sucesso} & Gerir corretamente o tempo de forma a abranger a diferentes variáveis que
	podem atrasar a entrega.\\
	\hline \textbf{Envolvimento} & Alto.\\
	\hline
\caption{Gerente de software do projeto}
\label{Gerente_de_software_do_projeto}
\end{longtable}

\textit{Principais necessidades dos Usuários ou dos Envolvidos}

\begin{longtable}{|m{2.75cm}|m{2cm}|m{3cm}|m{4cm}|m{4cm}|}
	\hline \textbf{Necessidade} & \textbf{Prioridade} & \textbf{Preocupações} & \textbf{Solução Atual} & \textbf{Solução
	 Proposta}\\
	\hline
Visualizar o gasto e a produção energética da casa.
&
Alta
&
Obter dados energéticos para visualização do balanço, já que, o objetivo do projeto “Casa sustentável” ter ser
sustentável, isto é, inclui gerar sua própria energia.
&
Não há solução.
&
Receber dados vindo da smart grid e dos geradores de energia e mostrar para o usuário através de gráficos.
\\
	\hline
Controlar aparelhos domésticos com a automação implementada.
&
Alta
&
O usuário tem que ter a habilidade de controlar todos os componentes automatizados da casa.
&
Há vários softwares no mercado que fazem esse tipo de monitoramento no mercado, mas nenhum deles tem integração com
todos os sistemas de monitoramento sustentável e de geração de energia que são necessários.
&
Desenvolver um sistema embarcado que consiga receber todos os dados de sensores e consiga controlar os aparelhos com
automação.
\\
	\hline
Acessibilidade
&
Alta
&
O software deve ter uma interface intuitiva e fácil de usar, já que o software engloba muitas funções de monitoramento
e controle.
&
Busca de técnicas de programação e criação de interface simples.
&
Analisar softwares presentes no mercado e fazer trabalho de pesquisa com protótipos com usuários reais.
\\
	\hline
%\caption{Principais necessidades dos Usuários ou dos Envolvidos}
%\label{Principais_necessidades_dos_Usuarios_ou_dos_Envolvidos}
\end{longtable}

\subsubsection{Ambiente do Usuário}

	O sistema será implementado e testado em um servidor local Linux, conectado através de ZWAVE juntamente com todos o
	pacote de software SACS.

	As restrições de uso incluem a compra do dispositivo com o software embutido e conexão com a internet na casa.

\subsubsection{Visão Geral do Sistema}

\subsubsubsection{Perspectiva do Sistema}

	O sistema tem como objetivo automatizar residências com a finalidade de trabalhar com termostatos, fechaduras,
	sistema de segurança, equipamentos de áudio e video, sistema de irrigação, portas de garagem, sensores ambientais,
	 câmeras, luzes, gerenciamento de energia, internet e redes relacionadas, controlador de piscinas e spa, integração
	  com radio frequência, integrado com robôs de limpeza, segurança dos dados, integração com telefones, fax, emails
	    e possuir interface de usuário.

\subsubsubsection{Suposições e Dependências}

	Para a utilização do sistema é suposto que a casa possua conexão com internet e vários sensores para controle do
	gasto de energia e sensor de fluxo de água.

	Caso haja necessidade de alterações no projeto, as alterações deverão ser feitas neste documento traduzindo de
	melhor maneira possível.

\subsubsection{Requisitos Funcionais do Sistema}

	Esta seção tem a função de informar de forma resumida todas as características e funcionalidades do sistema.

\begin{table}[!h]
\begin{tabular}{|l|l|c|c|}
	\hline
	\textbf{Requisito} & \textbf{Descrição} & \textbf{Prioridade} & \textbf{Dependência}\tabularnewline
	\hline
	\hline
	RF1 & Monitoramento energético & Alta & -\tabularnewline
	\hline
	RF2 & Ambientação & Médio & RF9, RF3\tabularnewline
	\hline
	RF3 & Gestão da informação & Alta & -\tabularnewline
	\hline
	RF4 & Controle de cortinas & Médio & RF3\tabularnewline
	\hline
	RF5 & Automatização de multimídia & Médio & RF3\tabularnewline
	\hline
	RF6 & Automatização hídrica & Médio & RF3\tabularnewline
	\hline
	RF7 & Automatização do Jardim & Baixa & RF3\tabularnewline
	\hline
	RF8 & Segurança & Alta & -\tabularnewline
	\hline
	RF9 & Detecção de presença & Médio & -\tabularnewline
	\hline
	RF10 & Monitoramento de humidade & Baixa & -\tabularnewline
	\hline
\end{tabular}
\caption{Requisitos Funcionais do Sistema}
\label{Requisitos_Funcionais_do_Sistema}
\end{table}

\subsubsection{Requisitos Não Funcionais do Sistema}

	Resumo em alto nível de outras características do sistema, tipicamente não funcionais.

\begin{table}[!h]
\begin{tabular}{|l|l|}
	\hline
	\textbf{Identificador} & \textbf{Recurso}\tabularnewline
	\hline
	\hline
	RNF1 & Taxa de transmissão de dados\tabularnewline
	\hline
	RNF2 & Estabilidade do sistema\tabularnewline
	\hline
	RNF3 & Integridade do sistema\tabularnewline
	\hline
	RNF4 & Monitoramento dos dados\tabularnewline
	\hline
	RNF5 & Precisão dos dados dos sensores\tabularnewline
	\hline
	RNF6 & Usabilidade do sistema\tabularnewline
	\hline
	RNF7 & Unidade central de processamento\tabularnewline
	\hline
	RNF8 & Monitoramento em tempo real\tabularnewline
	\hline
\end{tabular}
\caption{Requisitos Não Funcionais do Sistema}
\label{Requisitos_nao_Funcionais_do_Sistema}
\end{table}

\subsubsection{Procedência e Prioridades}

\begin{table}[!h]
\begin{tabular}{|l|l|}
	\hline
	\textbf{Prioridades} & \textbf{Descrição}\tabularnewline
	\hline
	\hline
	1 & Receber e mostrar dados e estatísticas de monitoramento. \tabularnewline
	\hline
	2 & Permitir o controle dos aparelhos automatizados. \tabularnewline
	\hline
	3 & Deve permitir o controle e visualização do sistema de segurança. \tabularnewline
	\hline
	4 & Deve mostrar quando algum componente estiver defeituoso. \tabularnewline
	\hline
\end{tabular}
\caption{Procedência e Prioridades}
\label{Procedencia_e_Prioridades}
\end{table}

%----------------------------------------------------%
%-------------------- Requisitos --------------------%
%----------------------------------------------------%
\section{Requisitos}
\label{sec:requisitos}

\subsection{Introdução}
\subsubsection{Finalidade}
	Esta seção concentra-se na coleta e na organização de todos os requisitos que envolvem o SACS. Descrevendo todos
	os requisitos de software para cada característica e funcionalidade de uma determinada ação do sistema.

\subsubsection{Escopo}

	O SACS, permite o usuário monitorar gastos de energia elétrica e consumo de água, controlar o sistema de segurança,
	visualizar a imagem das câmeras de segurança e controlar todo o sistema automatizado, incluindo sistema de
	jardinagem, luzes internas e outros aparelhos internos.

\subsubsection{Visão Geral}

	Esta seção servirá como base para a implantação do sistema SACS durante todo o desenvolvimento do projeto da Casa
	 Sustentável. Informações acerca das funcionalidades e características do SACS serão organizados e estarão
	 disponível neste documento afim de garantir que todas as necessidades sejam implementadas.

	Essa seção está dividido em: Descriçao Geral, Lista de Requisitos e Requisitos Específicos.

\subsection{Descrição Geral}

	O sistema SACS poderá lidar com termostatos, fechaduras, sistema de segurança, equipamentos de áudio e video, sistema
	de irrigação, portas de garagem, sensores ambientais, câmeras, luzes, gerenciamento de energia, internet e
	redes relacionadas, controlador de piscinas e spa, integração com radio frequência, integração com robôs de
	limpeza, segurança dos dados, integração com telefones, fax e emails, possuirá interface de usuário, integração
	 a consoles de jogos entre outros utilitários.

	Para a perfeita utilização do SACS é essencial que o sistema esteja em constante conectividade com a internet.
	Será necessário instalar sensores nas entradas das tomadas de energia para o controle dos gastos. A instalação
	de um núcleo de dados onde seja refrigerada e isolada se possível.

\subsection{Lista de Requisitos}
\begin{enumerate} %[I]
	\item \textbf{Requisitos Funcionais}
	\begin{enumerate} %[RF 1 -]
		\item Monitoramento energético
		\item Ambientação
		\item Gestão da informação
		\item Controle da disposição das cortinas
		\item Automatização de multimídia
		\item Automatização hídrica
		\item Automatização do jardim
		\item Segurança
		\item Decteção de presença
		\item Monitoramento da humidade
	\end{enumerate}

	\item \textbf{Requisitos Não Funcionais}
	\begin{enumerate} %[RNF 1 -]
		\item Taxa de transmissão de dados
		\item Estabilidade do sistema
		\item Integridade dos dados
		\item Monitoramento através da internet
		\item Precisão dos dados dos sensores
		\item Usabilidade do sistema
		\item Unidade central de processamento
		\item Monitoramento em tempo real
	\end{enumerate}
\end{enumerate}

\subsection{Requisitos Funcionais}
\subsubsection{Monitoramento energético}
\begin{itemize}

	\item \textbf{Descrição}
	O sistema deve medir o consumo energético em tempo real e gerar relatórios para o usuário.

	\item \textbf{Entradas}
Gastos de energia gerado por cada ponto de acesso de energia (tomada) da casa.

	\item \textbf{Saídas}
	Deve ser gerado relatórios diários, semanais e mensais, de gastos energéticos, de acordo com a requisição do usuário.

	\item \textbf{Processamento}
	Todo o consumo de energia gerado pela casa deve ser armazenado, de forma a ser acessível local e remotamente.

	Todo o consumo de energia gerado pela casa deve ser verificado, antes de
armazenado.

	\item \textbf{Manipulação de erros}
	Todos os dados do consumo energético deve ser verificado, e caso haja inconsistência ou erros nos dados, deve ser
	registrado no log do

\end{itemize}

\subsubsection{Ambientação}
\begin{itemize}
	\item \textbf{Descrição}
	As mudanças de temperatura e luzes devem influenciar na abertura das persianas, na força das luminárias e no HVAC
	para que o ambiente seja mantida constante e agradável.

	\item \textbf{Entradas}
	Sinais de sensores de luz e temperatura.

	\item \textbf{Saídas}
	Deve ser gerado um controle a fim de optimizar luz e temperatura fornecida pela casa.

	\item \textbf{Processamento}
	A luz e a temperatura da casa deve ser verificada e as modificações necessárias devem ser requisitadas a fim de
	manter o ambiente.

	\item \textbf{Manipulação de erros}
	Caso os dados não estejam acurados ou divergentes do esperados o sistema deve avisar o erro e não manipular mais
	nos objetos de automação de luz e temperatura.

\end{itemize}

\subsubsection{Controle da disposissão das cortinas}
\begin{itemize}

	\item \textbf{Descrição}
	O sistema a partir de um controle remoto ou ambiente pré configurado, deverá mudar a disposição das cortinas.

	\item \textbf{Entradas}
    Comandos enviados de um controle remoto.

	\item \textbf{Saídas}
    Manipulação da disposição das cortinas de acordo com o comando.

	\item \textbf{Processamento}
    A partir de dados pré-mapeados é feito o controle da disposição das cortinas referente ao que foi configurado.

	\item \textbf{Manipulação de erros}
	O sistema pode se deparar com a perda de um comando enviado a partir de um controle sendo necessário re-enviar o
	comando.

\end{itemize}

\subsubsection{Controle Multimídia}
\begin{itemize}
	\item \textbf{Descrição}
	O sistema deve garantir automatização de multimídia.

	\item \textbf{Entradas}
    Comandos enviados de um controle remoto.

	\item \textbf{Saídas}
	    Manipulação de audio, TV, projetor, dentre outros sistema multimídia presentes na casa.

	\item \textbf{Processamento}
	A partir de dados pré-mapeados é feito o controle de dispositivos de multimídia referente ao que foi configurado.

	\item \textbf{Manipulação de erros}
	O sistema pode se deparar com a perda de um comando enviado a partir de um controle sendo necessário re-enviar o
	comando.

\end{itemize}

\subsubsection{Monitoramento hídrico}
\begin{itemize}

	\item \textbf{Descrição}
	O sistema deve medir o consumo hídrico em tempo real e gerar relatórios para o usuário.

	\item \textbf{Entradas}
	Dados de consumo hídrico em tempo real.

	\item \textbf{Saídas}
	Relatórios diários, semanais e mensais, de gastos hídricos, de acordo com a requisição do usuário.

	\item \textbf{Processamento}
	Todo o consumo hídrico gerado pela casa deve ser verificado, antes de armazenado.

	Os dados do consumo hídrico devem ser processados para gerar um relatório contendo o consumo e o custo estimado da
	 conta de água.

	\item \textbf{Manipulação de erros}
	Todo o consumo hídrico gerado pela casa deve ser armazenado, após verificado, de forma segura em um hardware local.

\end{itemize}

\subsubsection{Controle de Jardim}
\begin{itemize}

	\item \textbf{Descrição}
	O sistema deve garantir automatização do jardim.

	\item \textbf{Entradas}
	Irrigação do jardim,de acordo com a necessidade de cada planta.

	\item \textbf{Saídas}
	Ser otimizado de modo a economizar água sem desperdicios.

	\item \textbf{Processamento}
	Ser colocado um tempo para irrigação de cada planta,vendo a necessidade de cada uma.

	\item \textbf{Manipulação de erros}
	Caso haja algum vazamento ou desperdicio de água o sistema deve avisar para não haver perdas de água desnecessarias.

\end{itemize}

\subsubsection{Controle de segurança de dados}
\begin{itemize}

	\item \textbf{Descrição}
	Os dados do sistema deverá garantir integridade, anonimato e consistência.

	\item \textbf{Entradas}
    Todos os dados do fluxo do sistema de forma.

	\item \textbf{Saídas}
	Dados criptografados e com seu respectivo hash.

	\item \textbf{Processamento}
	A partir dos dados de entrada, o sistema passa eles por dois algoritmos, um que criptografa e outros que tira sua
	hash (serve como indentidade).

	\item \textbf{Manipulação de erros}
	Todo dado após criptografado ao chegar ao destino é calculado novamente o hash e comparado com o anterior, antes
	de navegar pela rede e caso confirmado é de criptografado.

\end{itemize}

\subsubsection{Detectar presença}
\begin{itemize}

	\item \textbf{Descrição}
	O sistema a partir de sensores de presença deverá identificar qualquer movimento fora e dentro da casa.

	\item \textbf{Entradas}
	Alertas de detectação de movimento no perimetro.

	\item \textbf{Saídas}
	Informação da localização de onde foi detectado movimento e projeção da camera de video mais próximo deste ponto.

	\item \textbf{Processamento}
	Após a detecção de movimento, o sensor enviar para o sistema informações de onde foi detectado e libera a imagem
	da câmera mais próximo do local identificado.

	\item \textbf{Manipulação de erros}
	O sistema deverá lidar com a capacidade de ignorar movimentos de animais, ventos, folhas entre outros objetos.

\end{itemize}

\subsection{Requisitos Não Funcionais}

\subsubsection{Taxa de transmissão de dados}

\begin{itemize}

	\item \textbf{Performance}
	O sistema deve ser passível de cálculo da taxa de transmissão de dados de alta velocidade.

	\item \textbf{Confiabilidade}
	O sistema deve manter uma taxa de transmição de dados altas sem falha.

	\item \textbf{Disponibilidade}
	O sistema deve manter uma taxa de transmição de dados alta durante todo o periodo de funcionamento.

	\item \textbf{Segurança}
	A transmissão de dados deve ser inacessível a dispositivos não autorizados ou cadastrados.

	\item \textbf{Manutenibilidade}
	Em caso de falha, o sistema deve ser capaz de se recuperar em um tempo minimo a fim de nao deixar os aparelhos
	desamparados da trasmissão de dados.

	\item \textbf{Portabilidade}
	A transmissão de dados de alta velocidade deve abranger todos os dispositivos desejados da casa, além de ser
	capaz de se expandir para que a transmissão alcance novos dispositivos.

\end{itemize}

\subsubsection{Estabilidade sistemática}
\begin{itemize}

	\item \textbf{Performance}
	O sistema deve se manter estável durante situações críticas

	\item \textbf{Confiabilidade}
	O sistema deve se manter estável afim de manter o funcionamento energético da casa.

	\item \textbf{Disponibilidade}
	O funcionamento sistema deve ser manter estável 24hrs (horas) por dia.

	\item \textbf{Segurança}
	O sistema não deve ser vulnerável a falhas internas e externas que prejudiquem seu funcionamento por tempo integral.

	\item \textbf{Manutenibilidade}
	O sistema deve possuir uma manutenção pouco complexa e que não limite a eficiência e a eficâcia da mesma caso
	apresente alguma falha no seu funcionamento

	\item \textbf{Portabilidade}
	O sistema deve estar acessível a todos os dispositivos que regulem e mantenham o funcionamento estável do mesmo

\end{itemize}

\subsubsection{Integridade do sistema}
\begin{itemize}

	\item \textbf{Performance}
	Os dados do sistema deverá ter garantia de integridade, anonimato e consistência.

	\item \textbf{Confiabilidade}
	O sistema deve ser capaz de garantir a segurança de dados da casa garantindo que nenhum dado será apagado ou acessado.


	\item \textbf{Disponibilidade}
	O sistema deve manter a integridade dos dados durante todo o tempo de funcionamento.


	\item \textbf{Segurança}
	Os dados do sistema devem ser inacessíveis a dispositivos ou usuários não autorizados, preferencialmente atravez
	de criptografia dos dados e requerimento de login e senha do usuário.

	\item \textbf{Manutenibilidade}
	Em caso de falha, o sistema deve ser capaz de se recuperar em um tempo minimo a fim de nao deixar os aparelhos
	 desamparados da trasmissão de dados.

	\item \textbf{Portabilidade}
	A integridade dos dados deve se manter intácta para qualquer aparelho.

\end{itemize}

\subsubsection{Monitoramento através da internet}
\begin{itemize}

	\item \textbf{Performance}
	Deve ser garantido o monitoramento dos dados a partir da internet


	\item \textbf{Confiabilidade}
	O sistema deve manter uma transmição continua dos dados de segurança sempre que solicitado.

	\item \textbf{Disponibilidade}
	O sistema deve ter disponivel os dados sempre que solicitado.


	\item \textbf{Segurança}
	A transmissão de dados deve ser inacessível a dispositivos não autorizados.


	\item \textbf{Manutenibilidade}
	Em caso de falha, o sistema deve ser capaz de se recuperar em um tempo minimo a fim de não afetar os dados
	requeridos pelo usuário.


	\item \textbf{Portabilidade}
	O sistema deve ser acessivel de qualquer aparelho cadastrado.

\end{itemize}

\subsubsection{Alta precisão nos dados dos sensores}
\begin{itemize}

	\item \textbf{Performance}
	Os dados vindos dos sensores devem ser de alta precisão e velocidade, pois a partir dele, ações aconteceram no
	sistema.


	\item \textbf{Confiabilidade}
	O sistema de sensores devem ter alta confiabilidade nos seus dados, pois a partir deles ações seram tomadas
	externamente.


	\item \textbf{Disponibilidade}
	O monitoramento de dados depende diretamente da disponibilidade do sistema, sendo assim, o sensor deve ter um
	alto nivel de disponibilidade para fornecer esses dados sem comprometer o sistema.


	\item \textbf{Segurança}
	Os dados enviados pelos sensores devem ser mantidos dentro da rede segura e criptografada do sistema, impedindo
	monitoramento externo de terceiros.

	\item \textbf{Manutenibilidade}
	Em caso de falha, o sistema deve ser de fácil manutenção, podendo se resolver a maioria das falhas com procedimentos
	simples, evitando necessidade de um técnico com frequência.


	\item \textbf{Portabilidade}
	Os dados da casa devem ser acessiveis de qualquer aparelho conveniado com o sistema de qualquer lugar, atravez da
	internet.



\end{itemize}

\subsubsection{Usabilidade do sistema}
\begin{itemize}

	\item \textbf{Performance}
	O sistema deve ser de facil usabilidade para ser de fácil manuseio.


	\item \textbf{Confiabilidade}
	O sistema deve manter sua interface disponivel sempre que o usiuário solicitar.

	\item \textbf{Disponibilidade}
	O sistema deve ficar disponivel sempre que o usuário decidir manipular algo que exiga o uso do sistema.


	\item \textbf{Manutenibilidade}
	Em caso de falhas o sistema deve ser capaz de passar um feedback rapidamente para a empresa,para que seja feito a
	manutenção o mais rapido possivel para que o usuário não seja prejudicado.

\end{itemize}

\subsubsection{Central de processamento}
\begin{itemize}

	\item \textbf{Performance}
	O sistema deve possuir uma CPU (unidade central de processamento) capaz de satisfazer todos os processos executáveis
	 e significativos de entrada e saida desse sistema.

	\item \textbf{Confiabilidade}
	A CPU deve ser capaz de realizar as instruções do sistema afim de executa-las de uma maneira completa, consistente
	e correta.

	\item \textbf{Disponibilidade}
	O funcionamento do sistema demanda da disponibilidade integral de sua CPU dado que ela é um componente importante
	para o funcionamento desse sistema.


	\item \textbf{Segurança}
	A CPU deve manter-se estável caso seja imposta todo o fluxo de dados do sistema, afim de evitar falhas que
	comprometam o funcionamento do mesmo.


	\item \textbf{Manutenibilidade}
	A manutenção da CPU deve ser feita com facilidade e segurança afim de minimizar a complexidade da mesma, caso
	esteja operando com dificuldades no processamento, e com uma alta temperatura.

\end{itemize}

\subsubsection{Monitoramento em tempo real}
\begin{itemize}
	\item \textbf{Performance}
	O sistema deve ser fornecer dados de monitoramento da casa em tempo real de processamento

	\item \textbf{Confiabilidade}
	O sistema deve ser capaz de satisfazer o monitoramento em tempo real com o mínimo de interrupções possiveis, afim
	 de maximizar a segurança da SACS.


	\item \textbf{Disponibilidade}
	O sistema de processamento do monitoramento em tempo real deve estar disponivel em tempo integral por se tratar de
	 uma funcionalidade significativa para a segurança da casa.


	\item \textbf{Segurança}
	O sistema deve ser impassível a falhas de monitoramento, sendo estas externas e internas, que comprometam a
	segurança da casa.


	\item \textbf{Manutenibilidade}
	O sistema deve possuir uma manutenção com o mínimo de complexidade para não comprometer a sua segurança, caso
	esteja apresentando um nível fora do normal de interrupções ou esteja apresentando falhas no seu funcionamento.


	\item \textbf{Portabilidade}
	O sistema de monitoramento em tempo real deve estar disponível para ser visualizado por aplicações portadas para
	todos os dispositivos de visualização que o morador da casa tenha acesso.
\end{itemize}

\subsection{Componentes Comprados}

	O sistema possuirá dispositivos de hardware e software, que iram auxiliar nas funcionalidades de automatização da
	SACS, e que serão adquiridos pagando as devidas taxas referentes ao licenciamento e operabilidade destes dispositivos.


%-----------------------------------------------------------%
%-------------------- Seleção do Pacote --------------------%
%-----------------------------------------------------------%
\section{Seleção do Pacote}

\subsection{Pesquisa realizada sobre pacotes de automação}
\begin{table}[!h]
\begin{tabular}{|c|c|c|c|c|}
	\hline
	\textbf{Nome} & \textbf{Segurança} & \textbf{Iluminação} & \textbf{Irrigação} & \textbf{Multimídia}\tabularnewline
	\hline
	\hline
	Gdsautomacao & x & x & x & x\tabularnewline
	\hline
	Cynthron & x & x & - & x\tabularnewline
	\hline
	Smart home  & x & x & - & x\tabularnewline
	\hline
	Automatedliving  & x & x & ? & x\tabularnewline
	\hline
	\textcolor{red}{Homeseer} & \textcolor{red}{x} & \textcolor{red}{x} & \textcolor{red}{x} & \textcolor{red}{x}\tabularnewline
	\hline
	Control4 & x & x & - & -\tabularnewline
	\hline
	Creston  & x & x & x & x\tabularnewline
	\hline
	Vera  & x & x & x & -\tabularnewline
	\hline
	Stables conneted  & x & x & x & -\tabularnewline
	\hline
	Iris & x & x & x & -\tabularnewline
	\hline
	Savant & x & x & x & x\tabularnewline
	\hline
	SmartThings  & x & x & - & -\tabularnewline
	\hline
	Nexia & x & x & - & -\tabularnewline
	\hline
	Wallpad  & - & x & x & x\tabularnewline
	\hline
	AutomaticHouse & x & x & x & x\tabularnewline
	\hline
\end{tabular}
\caption{Pesquisa realizada sobre pacotes de automação}
\label{Pesquisa_realizada_sobre_pacotes_de_automacao}
\end{table}

\begin{table}[!h]
\begin{tabular}{|c|c|c|}
	\hline
	\textbf{Nome} & \textbf{Climatização} & \textbf{Câmera}\tabularnewline
	\hline
	\hline
	Gdsautomacao & x & x\tabularnewline
	\hline
	Cynthron & x & -\tabularnewline
	\hline
	Smart home  & x & x\tabularnewline
	\hline
	Automatedliving  & x & x\tabularnewline
	\hline
	\textcolor{red}{Homeseer} & \textcolor{red}{x} & \textcolor{red}{x}\tabularnewline
	\hline
	Control4 & - & x\tabularnewline
	\hline
	Creston  & x & x\tabularnewline
	\hline
	Vera  & - & x\tabularnewline
	\hline
	Stables conneted  & - & x\tabularnewline
	\hline
	Iris & - & x\tabularnewline
	\hline
	Savant & x & x\tabularnewline
	\hline
	SmartThings  & - & Third Party\tabularnewline
	\hline
	Nexia & - & x\tabularnewline
	\hline
	Wallpad  & x &  \tabularnewline
	\hline
	AutomaticHouse & x & x\tabularnewline
	\hline
\end{tabular}
\caption{Pesquisa realizada sobre pacotes de automação (Continuação)}
\label{Pesquisa_realizada_sobre_pacotes_de_automacao_continuacao}
\end{table}

%------------------------------------------------------------%
%-------------------- Termos Contratuais --------------------%
%------------------------------------------------------------%
\section{Termos Contratuais}

\subsection{Tipo de Contrato a ser Empregado}
	O contrato é estabelecido atráves da compra do pacote, onde o software está sendo disponibilizado,
	tendo seus termos de assinatura.

\subsection{Direitos de Distribuição, Uso e Propriedade do Software}
	Na realização da compra do pacote de software vem garantido toda licença de uso dos software em que ali estão.

%--------------------------------------------------------------------------%
%-------------------- Lista de Software a ser Entregue --------------------%
%--------------------------------------------------------------------------%
\section{Lista de Software a ser Entregue}

\begin{table}[!h]
\begin{tabular}{|l|l|}
	\hline
	\textbf{Software} & \textbf{Descrição}\tabularnewline
	\hline
	\hline
	HomeSeer Z-Wave & Luz e Tecnologias Primária\tabularnewline
	\hline
	HomeSeer ADIO100 & Monitoramento de Sensores Analógicos e Digitais\tabularnewline
	\hline
	HomeSeer Global Cache IR Con & Controle de Audio, Vídeo e Infra-vermelho\tabularnewline
	\hline
	HomeSeer Netcam & Captura de Fotos de Câmeras\tabularnewline
	\hline
	HomeSeer CURRENTCOST ENVI & Controle de Energia\tabularnewline
	\hline
	DD-WRT & Informações Sobre a Internet e a Rede\tabularnewline
	\hline
	HomeSeer Rain8 / Relay8 & Controle de Irrigação\tabularnewline
	\hline
	DirecTV & Controle de Mídia\tabularnewline
	\hline
	AquaConnect & Controle de Piscina e Spa\tabularnewline
	\hline
	HomeSeer DSC Security & Controle de Alarmes\tabularnewline
	\hline
	BLLock & Controle de Trava de Portas\tabularnewline
	\hline
	MyQ & Controle de Portão de Garagem\tabularnewline
	\hline
	HomeSeer HSPhone & Controle de Telefone, Correio de Voz e Lembretes\tabularnewline
	\hline
	HomeSeer HSTouch Server & Interface de Usuário (Android|IOS|Windows|Linux)\tabularnewline
	\hline
\end{tabular}
\caption{Lista de Software a ser Entregue}
\label{Lista_de_Software_a_ser_Entregue}
\end{table}

%----------------------------------------------------------------------------%
%-------------------- Critérios de Aceitação do Software --------------------%
%----------------------------------------------------------------------------%
\section{Critérios de Aceitação do Software}
	Dentre os críterios de aceitação do HomeSeer estão as licenças e todos os softwares que satisfaçam os requisitos
	exigidos para a automação da casa descritos na seção Objetivos.