\section{Metodologia}

	A metodologia que será usada no seguinte projeto é o Scrum, referente a metodologia ágil. A escolha dessa metodologia teve o intuito de tornar o trabalho mais fluido, dando mais dinamicidade e agilidade na confecção do trabalho. Outro benefício dessa metodologia é a facilidade de implementar mudanças no trabalho para atender as exigências do Dono do produto.
	
	O nosso Product Backlog será baseada no nosso escopo e na nossa \gls{EAP}. Nele haverá uma lista de funcionalidades que o nosso produto deve conter. Ele será de suma importância pois, a partir dele, será gerado o material para ser trabalhado nas Sprints.
	
	No decorrer do projeto haverá ciclos, Sprints, que terão a duração de uma semana. E em cada Sprint terá o desenvolvimento de atividades irão compor o produto final, o projeto da casa autossustentável. Elas começarão toda segunda feira com a Sprint Planning Meeting, em que nela será descrito os requisitos de maior importância e que devem estar na Sprint. Esses requisitos darão origem ao Sprint Backlog. E por fim elas terminam todo domingo. 
	
\subsection{Grupos e divisões}

	A equipe foi dividida em três grupos de pesquisa e um grupo de comunicação e documentação, para melhor organizar as pesquisas e o desenvolvimento do trabalho. Cada grupo possui um sub-gerente para concentrar a comunicação e não haver redundância de arquivos ou desentendimento de tarefas. 

\subsubsection{Gerente}
$\bullet$ \textbf{Guilherme Silva Lionço}\\

Foram selecionadas um grupo de pessoas para cuidar da parte de documentação e comunicação do time.
\subsubsection{Documentação e Comunicação}

$\bullet$ \textbf{Mateus M. F. Mendonça}

$\bullet$ Ricardo L. Canela\\

Os grupos de pesquisa que foram divididos são:

\subsubsection{Estrutura e Materiais}

$\bullet$ \textbf{Igor A. L. da Costa}

$\bullet$ Joao G. S. M. de Paula

$\bullet$ Kleiton N. Silva

$\bullet$ Klyssmann H. F. de Oliveira

$\bullet$ Marcelo M. de Oliveira

$\bullet$ Tulyane M. dos Santos

$\bullet$ Vitor N. A. Ribeiro

$\bullet$ Yuri S. Alves\\

\subsubsection{Eficiência Energética}

$\bullet$ \textbf{Jessica K. O. de Sousa}

$\bullet$ Barbara C. Ferro

$\bullet$ Ivson A. Rocha

$\bullet$ Jose M. M. F. Junior

$\bullet$ Stefany S. Aquino

$\bullet$ Talyta V. Cabral

$\bullet$ Walter L. Baldez\\

\subsubsection{Automação}

$\bullet$ \textbf{Gabriel S. Soares}

$\bullet$ Arnoldo T. M. Lima

$\bullet$ Flavio C. Paixão

$\bullet$ Itiane T. B. Almeida

$\bullet$ Lucas V. T. Brilhante

$\bullet$ Marcelo M. de Oliveira

$\bullet$ Victor B. Batalha\\

E foi criado um sub-grupo (de última hora) de \textbf{Software} dentro de \textbf{Automação}.

\subsubsection{Software}

$\bullet$ \textbf{Vitor N. A. Ribeiro}

$\bullet$ Flavio C. Paixão

$\bullet$ Lucas V. T. Brilhante

$\bullet$ Marcelo M. de Oliveira

$\bullet$ Mateus M. F. Mendonça\\
