\subsection{Introdução}
\subsubsection{Finalidade}
	Este documento de Especificação de Requisitos de Software concentra-se na coleta e na organização de todos os requisitos que envolvem o \gls{SACS}. Descrevendo todos os requisitos de software para cada característica e funcionalidade de uma determinada ação do sistema.

\subsubsection{Escopo}

	O \gls{SACS}, permite o usuário monitorar gastos de energia elétrica e consumo de água, controlar o sistema de segurança, visualizar a imagem das câmeras de segurança e controlar todo o sistema automatizado, incluindo sistema de jardinagem, luzes internas e outros aparelhos internos. O sistema também deve permitir a visualização da produção de energia das placas de energia solar e dos motores eólicos, além de mostrar o nível de reservatório de água da chuva.

\subsubsection{Visão Geral}

	Este documento servirá como base para a implantação do sistema SACS durante todo o desenvolvimento do projeto da Casa Sustentável. Informações acerca das funcionalidades e características do SACS serão organizados e estarão disponível neste documento afim de garantir que todas as necessidades sejam implementadas.
	
	Esse documento está dividido em: Descriçao Geral, Lista de Requisitos e Requisitos Específicos.

\subsection{Descrição Geral}

	O sistema \gls{SACS} lidará com termostatos, fechaduras, sistema de segurança, equipamentos de áudio e video, sistema de irrigação, portas de garagem, sensores ambientais, câmeras, luzes, gerenciamento de energia, internet e redes relacionadas, controlador de piscinas e spa, integração com radio frequência, integração com robôs de limpeza, segurança dos dados, integração com telefones, fax e emails, possuirá interface de usuário, integração a consoles de jogos entre outros utilitários.

	Para a perfeita utilização do \gls{SACS} é essencial que o sistema esteja em constante conectividade com a internet. Será necessário instalar sensores nas entradas das tomadas de energia para o controle dos gastos. A instalação de um núcleo de dados onde seja refrigerada e isolada se possível. 

\subsection{Lista de Requisitos}
\begin{enumerate}[I]
	\item \textbf{Requisitos Funcionais}
	\begin{enumerate}[RF 1 -]
		\item Monitoramento energético
		\item Ambientação
		\item Gestão da informação
		\item Controle da disposição das cortinas
		\item Automatização de multimídia
		\item Automatização hídrica
		\item Automatização do jardim
		\item Segurança
		\item Decteção de presença
		\item Monitoramento da humidade
	\end{enumerate}

	\item \textbf{Requisitos Não Funcionais}
	\begin{enumerate}[RNF 1 -]
		\item Taxa de transmissão de dados
		\item Estabilidade do sistema
		\item Integridade dos dados
		\item Monitoramento através da internet
		\item Precisão dos dados dos sensores
		\item Usabilidade do sistema
		\item Unidade central de processamento
		\item Monitoramento em tempo real
	\end{enumerate}
\end{enumerate}

\subsection{Requisitos Funcionais}
\subsubsection{Monitoramento energético}
\begin{itemize}

	\item \textbf{Descrição}
	O sistema deve medir o consumo energético em tempo real e gerar relatórios para o usuário.

	\item \textbf{Entradas}
Gastos de energia gerado por cada ponto de acesso de energia (tomada) da casa.

	\item \textbf{Saídas}
	Deve ser gerado relatórios diários, semanais e mensais, de gastos energéticos, de acordo com a requisição do usuário.

	\item \textbf{Processamento}
	Todo o consumo de energia gerado pela casa deve ser armazenado, de forma a ser acessível local e remotamente.

	Todo o consumo de energia gerado pela casa deve ser verificado, antes de
armazenado.

	\item \textbf{Manipulação de erros}
	Todos os dados do consumo energético deve ser verificado, e caso haja inconsistência ou erros nos dados, deve ser registrado no log do

\end{itemize}

\subsubsection{Ambientação}
\begin{itemize}
	\item \textbf{Descrição}
	As mudanças de temperatura e luzes devem influenciar na abertura das persianas, na força das luminárias e no HVAC para que o ambiente seja mantida constante e agradável.

	\item \textbf{Entradas}
	Sinais de sensores de luz e temperatura.

	\item \textbf{Saídas}
	Deve ser gerado um controle a fim de optimizar luz e temperatura fornecida pela casa.

	\item \textbf{Processamento}
	A luz e a temperatura da casa deve ser verificada e as modificações necessárias devem ser requisitadas a fim de manter o ambiente.

	\item \textbf{Manipulação de erros}
	Caso os dados não estejam acurados ou divergentes do esperados o sistema deve avisar o erro e não manipular mais nos objetos de automação de luz e temperatura.

\end{itemize}

\subsubsection{Controle da disposissão das cortinas}
\begin{itemize}

	\item \textbf{Descrição}
	O sistema a partir de um controle remoto ou ambiente pré configurado, deverá mudar a disposição das cortinas.

	\item \textbf{Entradas}
    Comandos enviados de um controle remoto.

	\item \textbf{Saídas}
    Manipulação da disposição das cortinas de acordo com o comando.

	\item \textbf{Processamento}
    A partir de dados pré-mapeados é feito o controle da disposição das cortinas referente ao que foi configurado.

	\item \textbf{Manipulação de erros}
	O sistema pode se deparar com a perda de um comando enviado a partir de um controle sendo necessário re-enviar o comando.

\end{itemize}

\subsubsection{Controle Multimídia}
\begin{itemize}
	\item \textbf{Descrição}
	O sistema deve garantir automatização de multimídia.

	\item \textbf{Entradas}
    Comandos enviados de um controle remoto.

	\item \textbf{Saídas}
	    Manipulação de audio, TV, projetor, dentre outros sistema multimídia presentes na casa.

	\item \textbf{Processamento}
	A partir de dados pré-mapeados é feito o controle de dispositivos de multimídia referente ao que foi configurado.

	\item \textbf{Manipulação de erros}
	O sistema pode se deparar com a perda de um comando enviado a partir de um controle sendo necessário re-enviar o comando.

\end{itemize}

\subsubsection{Monitoramento hídrico}
\begin{itemize}

	\item \textbf{Descrição}
	O sistema deve medir o consumo hídrico em tempo real e gerar relatórios para o usuário.

	\item \textbf{Entradas}
	Dados de consumo hídrico em tempo real.

	\item \textbf{Saídas}
	Relatórios diários, semanais e mensais, de gastos hídricos, de acordo com a requisição do usuário.

	\item \textbf{Processamento}
	Todo o consumo hídrico gerado pela casa deve ser verificado, antes de armazenado.
	
	Os dados do consumo hídrico devem ser processados para gerar um relatório contendo o consumo e o custo estimado da conta de água.

	\item \textbf{Manipulação de erros}
	Todo o consumo hídrico gerado pela casa deve ser armazenado, após verificado, de forma segura em um hardware local.

\end{itemize}

\subsubsection{Controle de Jardim}
\begin{itemize}

	\item \textbf{Descrição}
	O sistema deve garantir automatização do jardim.

	\item \textbf{Entradas}
	Irrigação do jardim,de acordo com a necessidade de cada planta.

	\item \textbf{Saídas}
	Ser otimizado de modo a economizar água sem desperdicios.

	\item \textbf{Processamento}
	Ser colocado um tempo para irrigação de cada planta,vendo a necessidade de cada uma.

	\item \textbf{Manipulação de erros}
	Caso haja algum vazamento ou desperdicio de água o sistema deve avisar para não haver perdas de água desnecessarias.

\end{itemize}

\subsubsection{Controle de segurança de dados}
\begin{itemize}

	\item \textbf{Descrição}
	Os dados do sistema deverá garantir integridade, anonimato e consistência.

	\item \textbf{Entradas}
    Todos os dados do fluxo do sistema de forma.

	\item \textbf{Saídas}
	Dados criptografados e com seu respectivo hash.

	\item \textbf{Processamento}
	A partir dos dados de entrada, o sistema passa eles por dois algoritmos, um que criptografa e outros que tira sua hash (serve como indentidade).

	\item \textbf{Manipulação de erros}
	Todo dado após criptografado ao chegar ao destino é calculado novamente o hash e comparado com o anterior, antes de navegar pela rede e caso confirmado é de criptografado.

\end{itemize}

\subsubsection{Detectar presença}
\begin{itemize}

	\item \textbf{Descrição}
	O sistema a partir de sensores de presença deverá identificar qualquer movimento fora e dentro da casa.

	\item \textbf{Entradas}
	Alertas de detectação de movimento no perimetro.

	\item \textbf{Saídas}
	Informação da localização de onde foi detectado movimento e projeção da camera de video mais próximo deste ponto.

	\item \textbf{Processamento}
	Após a detecção de movimento, o sensor enviar para o sistema informações de onde foi detectado e libera a imagem da câmera mais próximo do local identificado.

	\item \textbf{Manipulação de erros}
	O sistema deverá lidar com a capacidade de ignorar movimentos de animais, ventos, folhas entre outros objetos.

\end{itemize}

\subsubsection{Monitoramento do solo}
\begin{itemize}

	\item \textbf{Descrição}
	O sistema deve medir a umidade do solo em tempo real, gerar relatórios para o usuário e 

	\item \textbf{Entradas}
	Dados da umidade do solo em tempo real.

	\item \textbf{Saídas}
	Relatórios diários, semanais e mensais, da umidade do solo, de acordo com a requisição do usuário.

	\item \textbf{Processamento}
	Os dados da umidade do solo devem ser armazenados de forma segura em um hardware local, e processados para gerar um relatório contendo a variação ao longo do dia.

	O usuario deve ser alertado caso a umidade atinja um nível crítico.

	\item \textbf{Manipulação de erros}
	Todos os dados da medição da umidade do solo deve ser verificado, antes de armazenado.

\end{itemize}

\subsection{Requisitos Não Funcionais}

\subsubsection{Taxa de transmissão de dados}

\begin{itemize}

	\item \textbf{Performance}
	O sistema deve ser passível de cálculo da taxa de transmissão de dados de alta velocidade. 

	\item \textbf{Confiabilidade}
	O sistema deve manter uma taxa de transmição de dados altas sem falha.

	\item \textbf{Disponibilidade}
	O sistema deve manter uma taxa de transmição de dados alta durante todo o periodo de funcionamento.

	\item \textbf{Segurança}
	A transmissão de dados deve ser inacessível a dispositivos não autorizados ou cadastrados.

	\item \textbf{Manutenibilidade}
	Em caso de falha, o sistema deve ser capaz de se recuperar em um tempo minimo a fim de nao deixar os aparelhos desamparados da trasmissão de dados.

	\item \textbf{Portabilidade}
	A transmissão de dados de alta velocidade deve abranger todos os dispositivos desejados da casa, além de ser capaz de se expandir para que a transmissão alcance novos dispositivos.

\end{itemize}

\subsubsection{Estabilidade sistemática}
\begin{itemize}

	\item \textbf{Performance}
	O sistema deve se manter estável durante situações críticas

	\item \textbf{Confiabilidade}
	O sistema deve se manter estável afim de manter o funcionamento energético da casa.

	\item \textbf{Disponibilidade}
	O funcionamento sistema deve ser manter estável 24hrs (horas) por dia.

	\item \textbf{Segurança}
	O sistema não deve ser vulnerável a falhas internas e externas que prejudiquem seu funcionamento por tempo integral.

	\item \textbf{Manutenibilidade}
	O sistema deve possuir uma manutenção pouco complexa e que não limite a eficiência e a eficâcia da mesma caso apresente alguma falha no seu funcionamento

	\item \textbf{Portabilidade}
	O sistema deve estar acessível a todos os dispositivos que regulem e mantenham o funcionamento estável do mesmo
	
\end{itemize}

\subsubsection{Integridade do sistema}
\begin{itemize}

	\item \textbf{Performance}
	Os dados do sistema deverá ter garantia de integridade, anonimato e consistência. 

	\item \textbf{Confiabilidade}
	O sistema deve ser capaz de garantir a segurança de dados da casa garantindo que nenhum dado será apagado ou acessado.


	\item \textbf{Disponibilidade}
	O sistema deve manter a integridade dos dados durante todo o tempo de funcionamento.


	\item \textbf{Segurança}
	Os dados do sistema devem ser inacessíveis a dispositivos ou usuários não autorizados, preferencialmente atravez de criptografia dos dados e requerimento de login e senha do usuário.

	\item \textbf{Manutenibilidade}
	Em caso de falha, o sistema deve ser capaz de se recuperar em um tempo minimo a fim de nao deixar os aparelhos desamparados da trasmissão de dados.
	
	\item \textbf{Portabilidade}
	A integridade dos dados deve se manter intácta para qualquer aparelho.

\end{itemize}

\subsubsection{Monitoramento através da internet}
\begin{itemize}

	\item \textbf{Performance}
	Deve ser garantido o monitoramento dos dados a partir da internet
 

	\item \textbf{Confiabilidade}
	O sistema deve manter uma transmição continua dos dados de segurança sempre que solicitado.  

	\item \textbf{Disponibilidade}
	O sistema deve ter disponivel os dados sempre que solicitado.



	\item \textbf{Segurança}
	A transmissão de dados deve ser inacessível a dispositivos não autorizados.


	\item \textbf{Manutenibilidade}
	Em caso de falha, o sistema deve ser capaz de se recuperar em um tempo minimo a fim de não afetar os dados requeridos pelo usuário.


	\item \textbf{Portabilidade}
	O sistema deve ser acessivel de qualquer aparelho cadastrado.

\end{itemize}

\subsubsection{Alta precisão nos dados dos sensores}
\begin{itemize}

	\item \textbf{Performance}
	Os dados vindos dos sensores devem ser de alta precisão e velocidade, pois a partir dele, ações aconteceram no sistema. 


	\item \textbf{Confiabilidade}
	O sistema de sensores devem ter alta confiabilidade nos seus dados, pois a partir deles ações seram tomadas externamente.


	\item \textbf{Disponibilidade}
	O monitoramento de dados depende diretamente da disponibilidade do sistema, sendo assim, o sensor deve ter um alto nivel de disponibilidade para fornecer esses dados sem comprometer o sistema.


	\item \textbf{Segurança}
	Os dados enviados pelos sensores devem ser mantidos dentro da rede segura e criptografada do sistema, impedindo monitoramento externo de terceiros.

	\item \textbf{Manutenibilidade}
	Em caso de falha, o sistema deve ser de fácil manutenção, podendo se resolver a maioria das falhas com procedimentos simples, evitando necessidade de um técnico com frequência.


	\item \textbf{Portabilidade}
	Os dados da casa devem ser acessiveis de qualquer aparelho conveniado com o sistema de qualquer lugar, atravez da internet.



\end{itemize}

\subsubsection{Usabilidade do sistema}
\begin{itemize}

	\item \textbf{Performance}
	O sistema deve ser de facil usabilidade para ser de fácil manuseio.


	\item \textbf{Confiabilidade}
	O sistema deve manter sua interface disponivel sempre que o usiuário solicitar.

	\item \textbf{Disponibilidade}
	O sistema deve ficar disponivel sempre que o usuário decidir manipular algo que exiga o uso do sistema.


	\item \textbf{Manutenibilidade}
	Em caso de falhas o sistema deve ser capaz de passar um feedback rapidamente para a empresa,para que seja feito a manutenção o mais rapido possivel para que o usuário não seja prejudicado.

\end{itemize}

\subsubsection{Central de processamento}
\begin{itemize}

	\item \textbf{Performance}
	O sistema deve possuir uma CPU (unidade central de processamento) capaz de satisfazer todos os processos executáveis e significativos de entrada e saida desse sistema.

	\item \textbf{Confiabilidade}
	A CPU deve ser capaz de realizar as instruções do sistema afim de executa-las de uma maneira completa, consistente e correta. 

	\item \textbf{Disponibilidade}
	O funcionamento do sistema demanda da disponibilidade integral de sua CPU dado que ela é um componente importante para o funcionamento desse sistema.


	\item \textbf{Segurança}
	A CPU deve manter-se estável caso seja imposta todo o fluxo de dados do sistema, afim de evitar falhas que comprometam o funcionamento do mesmo.


	\item \textbf{Manutenibilidade}
	A manutenção da CPU deve ser feita com facilidade e segurança afim de minimizar a complexidade da mesma, caso esteja operando com dificuldades no processamento, e com uma alta temperatura.

\end{itemize}

\subsubsection{Monitoramento em tempo real}
\begin{itemize}

	\item \textbf{Performance}
	O sistema deve ser fornecer dados de monitoramento da casa em tempo real de processamento

	\item \textbf{Confiabilidade}
	O sistema deve ser capaz de satisfazer o monitoramento em tempo real com o mínimo de interrupções possiveis, afim de maximizar a segurança da SACS.


	\item \textbf{Disponibilidade}
	O sistema de processamento do monitoramento em tempo real deve estar disponivel em tempo integral por se tratar de uma funcionalidade significativa para a segurança da casa.


	\item \textbf{Segurança}
	O sistema deve ser impassível a falhas de monitoramento, sendo estas externas e internas, que comprometam a segurança da casa.


	\item \textbf{Manutenibilidade}
	O sistema deve possuir uma manutenção com o mínimo de complexidade para não comprometer a sua segurança, caso esteja apresentando um nível fora do normal de interrupções ou esteja apresentando falhas no seu funcionamento.


	\item \textbf{Portabilidade}
	O sistema de monitoramento em tempo real deve estar disponível para ser visualizado por aplicações portadas para todos os dispositivos de visualização que o morador da casa tenha acesso.


\end{itemize}

\subsection{Componentes Comprados}
	
	O sistema possuirá dispositivos de hardware e software, que iram auxiliar nas funcionalidades de automatização da SACS, e que serão adquiridos pagando as devidas taxas referentes ao licenciamento e operabilidade destes dispositivos.

